\chapter{Conclusions and Future Work}
\label{cap:conclusions}

\chapterquote{That's just how it is: when you get over one milestone, there's another, bigger one}{Allan Holdsworth}

We have developed a framework for Elixir code verification across several areas.
First, with the integration of \acrshort{smt} solvers compliant with the SMT-LIB
standard in Elixir. Second, with a verification \gls{ir} suitable to express
Elixir terms, their dynamically typed nature and some ad-hoc operator behaviors
such as the short-circuit evaluation of the strict logical connectives
\verb|and| and \verb|or|. Finally, with a formal language that corresponds to 
Elixir code with ghost verification expressions that is translated into our
\gls{ir}, and an early implementation for offering all of this within a simple 
Elixir \gls{api}.

Although we delimited the scope of this project at the beginning, it has ended
up being broader than we thought, and we have left several improvement tasks on
the way.

Regarding the \acrshort{smt} solver integration for Elixir, it can be extended
to support more of the SMT-LIB standard and to offer more out-of-the-box solver 
support apart from Z3. It would be also interesting to define the small step 
operational semantics of the L0 language and its correspondence with the big
step one in order to better study its implementation.

The built-in specifications and types for our verification \gls{ir} to model the
Elixir semantics can be extended by adding support for more of the built-in
Elixir value types and functions, even by adding new L1 expressions and
statements to support features that could not be modelled with the current
syntax, for example, higher-order functions. Also, questions can arise regarding
how well our model adheres to the Elixir semantics, for which having a
definition of the Elixir operational semantics would be required.

Besides, the L2 language can also be extended to support more features of the
Elixir language, such as exceptions and the Elixir \verb|pin| operator, and to
offer more verification features. The unfolding of user-defined functions has
been also left as future work, and the resulting tool in general is in an early
minimal stage.

As we have mentioned, regarding the non-goals of this project, it would be
interesting to extend the ideas behind it to also verify concurrent Elixir
constructs such as \verb|send| and \verb|receive|, which would not be a small
endeavor at all, and, prior to that, to achieve total verification by taking
termination into account.

It has been pleasant to implement our ideas in a high-level programming language
such as Elixir, whose functional programming principles allowed us to write code
that usually resembles our formalization concisely, although sometimes it is
hard to develop non-toy programs with dynamically typed systems and try to not 
suffer too much when finding bugs or refactoring. Maybe projects like this reach
at some point the maturity to alleviate that even in dynamically typed 
programming languages.
